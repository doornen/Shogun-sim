\documentclass[12pt]{book}
\usepackage{amssymb, amscd}
\usepackage[latin1]{inputenc}
\usepackage[T1]{fontenc}
%\usepackage[ngerman]{babel}
\usepackage{amsmath}
%\noindent
\setlength{\parindent}{0pt}
 
\newcommand{\versionPackage} {Geant4Shogun.1.2.1.tar.gz}
\newcommand{\versionROOT} {5.34/34}
\newcommand{\versionGEANT} {g4.10.2}

\begin{document}

\pagestyle{headings}

\author{Pieter Doornenbal}
\date{\today} 
\title{Manual of a GEANT4 Simulation Code for $\gamma$-Ray Detectors used
  in the RIKEN-RIBF Facility}

\maketitle
\tableofcontents

\chapter{Introduction}
\sloppy
This manual describes briefly the GEANT4~\cite{GEANT4} simulation code of the $\gamma$-ray 
spectrometers Dali2~\cite{takeuchi:2014:nima} and SHOGUN, which are or will be employed at the 
Radioactive Isotope Beam Factory (RIBF).
The aim of the simulation code is to obtain reliable values for the $\gamma$-ray detection efficiency
and energy resolution of a chosen detector geometry under realistic experimental conditions
for the secondary beams. Also effects on the $\gamma$-ray line-shape from the target thickness and
lifetimes of the excited states can be studied. In the following description the necessary steps 
to perform a simulation and the options are discussed.\hfill{}
\linebreak{}
\linebreak{}
A typical simulation is divided into three steps, namely:
\begin{itemize}
\item The \textit{Event Generator}, in which a heavy ion beam strikes on a target and emits
  $\gamma$-rays. Incoming beam and outgoing beam may be different, i.e. fragmentation reactions
  are covered in the simulation.
\item The \textit{Event Builder} simulates the $\gamma$-ray detection response and uses the 
  first step as input values.
\item The \textit{Reconstructor} performs the necessary Doppler correction of the detected
  $\gamma$-rays. Here, the analysis of the observed $\gamma$-rays can be performed by the
  user.
\end{itemize}
The motivation for partitioning the simulation into these three steps is that they are in principle
independent from each other. Thus, when changing for instance the detector geometry, the first step
does not have to be re-simulated and can be used again. 
Since the first step is very time consuming (especially with thick targets), the whole
simulation process is thereby accelerated.\hfill{}
\linebreak{}
\linebreak{}
Prior to explaining the three steps in detail, a few notes on the software necessary to run the simulation need
to be mentioned. The following software is required to be installed on your computer:
\begin{itemize}
\item The GEANT4 framework. The simulation code was tested only
  for {\versionGEANT}. It can be downloaded from:\hfill{} 
  \linebreak {\ttfamily http://geant4.cern.ch/}.
 % \linebreak
 % An installation guide for Linux and Windows machines is given in:\hfill{} 
 % \linebreak {\ttfamily http://geant4.slac.stanford.edu/installation/}.
%\item The CLHEP class library. Details are given in the previously mentioned installation guide and
%  under:\hfill{} 
%  \linebreak {\ttfamily http://proj-clhep.web.cern.ch/proj-clhep/}.
\item The ROOT~\cite{ROOT} framework. It can be downloaded from:\hfill{} 
  \linebreak {\ttfamily http://root.cern.ch/}.
  \linebreak
  The simulation code  was tested successfully with ROOT version \versionROOT. 
\end{itemize}
 
The simulation package has to be unpacked with the command:\hfill{}
\linebreak
\linebreak {\ttfamily tar -zxf \versionPackage.}
\linebreak
\linebreak
This will create the sub-folders {\ttfamily EventGenerator}, 
{\ttfamily EventBuilder}, and {\ttfamily Reconstructor}. 
These sub-folders contain the
the Event Generator, the Event Builder, and the Reconstructor, respectively, and are now covered in detail.
%To setup your environment on the machine RIBF00, you may type:\hfill{}
%\linebreak
%\linebreak {\ttfamily source setup.sh}
%\linebreak
%\linebreak
%You can put the lines into your {\ttfamily .bashrc} file or, if your installation directories are different,
%modify the locations accordingly. 
The manual will close with remarks on things that need improvement and/or need to be implemented. Note that the
simulation package uses a left-handed system with the positive z-axis being the beam direction.

\chapter{The Event Generator}

Go the the sub-folder {\ttfamily EventGenerator}.
Open the file {\ttfamily GNUMakefile} and make sure that the variables {\ttfamily G4INSTALL}
and {\ttfamily CLHEP\_BASE\_DIR} are set according to the installation location on your computer.
Compile the Event Generator with:\hfill{}
\linebreak
\linebreak {\ttfamily make all}
\linebreak
\linebreak
It will create the EventGenerator. Please ignore all the warning messages.

\section{The Input File(s)}

To change the parameters of the Event Generator, open the file 
{\ttfamily ./input/EventGenerator.in}. The file contains keywords, which can be used
in free format and change the default values of the simulation. The available keywords and their
parameters are listed in Tab.\ref{tab:EVENTGENERATORKEY}. In this table, $s$, $i$, and $f$ are used
for input string, integer, and floating point values, respectively. Note that in the current version
the simulation code is very restrictive, that is, comments and not existing input keywords in this file
will cause the program to terminate.

\begin{table}
  \centering
  \label{tab:EVENTGENERATORKEY}
  %\begin{ruledtabular}
  \begin{tabular}{|l||c|}
    \hline
    Keyword & Parameters \\
    \hline
    \hline
    BEAMISOTOPE             & $i$ $i$ $i$     \\
    BEAMENERGY              & $f$ $f$         \\
    BEAMPOSITION            & $f$ $f$ $f$ $f$ \\
    BEAMANGLE               & $f$ $f$ $f$ $f$ \\
    TARGET                  & $i$ $f$ $f$ $f$ \\
    TARGETANGULARBROADENING & $i$ $f$         \\
    MASSCHANGE              & $i$ $i$         \\
    BORREL                  & $i$ $f$         \\
    GOLDHABER               & $i$ $f$         \\
    GAMMAINPUT              & $s$             \\
    THETARANGE              & $f$ $f$         \\
    NUMBEROFEVENTS          & $i$             \\
    DEFAULTCUTVALUE         & $f$             \\
    OUTPUTFILE              & $s$             \\
    DEDXTABLE               & $i$ $s$ $s$     \\
    DISTRIBUTIONTYPE        & $i$             \\
    ATOMICBG                & $s$ $s$ $f$ $f$ \\
    END                     &                 \\
    \hline
  \end{tabular}
\end{table}

The parameters are now discussed subsequently (if not specified differently, 
the units are cm and degrees, respectively):\hfill{}
\begin{itemize}
\item BEAMISOTOPE $A_{P}$ $Z_{P}$ $Q_{P}$ \hfill{} \linebreak
  Contains the information on the type of the projectile $P$ in the order 
  mass $A_{P}$, the element number $Z_{P}$ and the charge state $Q_{P}$.
\item BEAMENERGY $E_{P}$ $\Delta E(FWHM)_{P}$\hfill{} \linebreak
  gives the total energy of the projectile 
  ( $E_{P}$ before striking on the target) 
  and the width of the energy distribution ($\Delta E(FWHM)_{P}$) in MeV/$u$.
\item BEAMPOSITION X FWHM$_{X}$ Y FWHM$_{Y}$\hfill{} \linebreak 
  Position of the projectile before impinging on the target. 
\item BEAMANGLE $\vartheta_{P}$ $\Delta (FWHM) \vartheta_{P}$ $\varphi_{P_{min}}$ $\varphi_{P_{max}}$
  \hfill{} \linebreak
  Angle of the incoming projectile. The distribution for $\Delta (FWHM) \vartheta_{P}$ is Gaussian, while the 
  distribution between $\varphi_{P_{min}}$ and $\varphi_{P_{max}}$ is flat.
\item TARGET Type Size$_{X}$ Size$_{Y}$ Thickness$_{Z}$\hfill{} \linebreak 
  This line specifies the target material and its dimensions. 
  The Type is 1 for Au, 2 for Be, 3 for C, 4 for Fe, 5 for Pb, and 6 for LH$_{2}$. 
  As density always the ``standard'' value for a solid state is given.
  The value for Thickness$_{Z}$ is given in mg/cm$^{2}$.
\item TARGETANGULARBROADENING Option$_{ang.}$ $\Delta (FWHM) \vartheta_{target}$\hfill{} \linebreak 
  Angular Broadening caused by the reaction in the target. If Option$_{ang.}$=1,
  this option is considered and the broadening is a Gaussian distribution defined by $\Delta (FWHM) \vartheta_{target}$.
\item MASSCHANGE $\Delta A$ $\Delta Z$\hfill{} \linebreak
\item BORREL: Option$_{Borrel}$ $B_{n}$\hfill{} \linebreak
  Velocity shift for a fragmentation process. If Option$_{Borrel}$=1, the velocity shift is calculated according
  to the formula from Ref.~\cite{BORREL}:
  \begin{equation}
    \frac{v_{f}}{v_{p}} = \sqrt{1-\frac{B_{n}(A_{P}-A_{F})}{A_{P}E_{F}}},
  \end{equation}
  where the index $F$ stands for the fragment,  $v$ is the velocity, 
  and $B_n$ the binding energy (in MeV) per ablated nucleon.
\item GOLDHABER Option$_{Goldhaber}$ $\sigma_{0}$\hfill{} \linebreak
  Parallel momentum distribution produced in a fragmentation process. If Option$_{Goldhaber}$=1, the momentum 
  distribution is calculated according to: 
  \begin{equation}
    \sigma_{||}=\sigma_{0}^{2} \frac{A_{F}(A_{P}-A_{F})}{A_{P}-1},
  \end{equation}
  where $\sigma_{0}$ is given in MeV/c.
\item GAMMAINPUT $File_{\gamma-in}$\hfill{} \linebreak
  The Filename specifies the location of the level and decay scheme to be simulated.
\item THETARANGE $\vartheta_{\gamma_{min}}$ $\vartheta_{\gamma_{max}}$\hfill{} \linebreak
  Polar angular range in the moving frame of the $\gamma$-rays that should be included in the simulation. 
  If your detectors cover only extreme forward angles, $\vartheta_{\gamma_{min}}$ can be set to 0 and 
  $\vartheta_{\gamma_{max}}$ to 90, thereby reducing the simulation time and file-size by a factor of two.
\item NUMBEROFEVENTS $N_{events}$\hfill{} \linebreak
  $N_{events}$ gives the number of reactions to be simulated.
\item DEFAULTCUTVALUE $L_{cut}$\hfill{} \linebreak
  $L_{cut}$ specifies the default cut value in mm used in the simulation. Visit the GEANT4 web pages for more information.
\item OUTPUFILENAME $File_{out}$\hfill{} \linebreak
  Specifies the location of the output file name.
\item DEDXTABLE $Option_{dEdX}$ $File_{P}$ $File_{E}$\hfill{} \linebreak
  Instead of letting GEANT4 calculate the energy loss of projectile and ejectile (the fragment), if Option$_{dEdX}$=1
  the energy loss can be calculated much faster (for thick targets) according to energy loss tables. 
  $File_{P}$ and $File_{E}$ specify the location of these tables for projectile and ejectile, respectively.
\item DISTRIBUTIONTYPE $\vartheta_{\gamma}$\hfill{} \linebreak
  Angular distribution of $\gamma$-rays along beam axis. $\vartheta_{\gamma}=0$ corresponds to uniform distribution.
  Only angular distributions of quadrupole transitions from the $2^{+}$ state can be simulated. 
  For the different $m=\pm I_{i}$ substates, $\vartheta_{\gamma}$ has to be put as follows: 
  \linebreak
    \linebreak
  {\ttfamily
2200: J,M 2,2-> 0,0      \linebreak
2100: J,M 2,1->0,0     \linebreak
2000: J,M 2,0 -> 0,0    \linebreak
    }
  \linebreak
    \linebreak

\item ATOMICBG $Option_{ATOMICBG}$ $FileName$ $Spectrumname$ $\sigma_{bg}$ $N_{\gamma}$\hfill{} \linebreak
  Option to include atomic background from a 2D ROOT histogram. The total cross section of the
  atomic background has to be specified in $\sigma_{bg}$ in mbarn and the total number of $\gamma$-rays produced
  per incident particle, $N_{\gamma}$, has to be specified. A separate program to calculate the
  anticipated atomic background, abkg, is provided. More details are given in a Sec.~\ref{sec:abkg}.
\item END\hfill{} \linebreak
  This keyword will end the scan of the input file and can be set at any line within the input file.
\end{itemize}

\subsection{Input of the Level Scheme}

The variable $File_{\gamma-in}$ specifies the location of the decay scheme to be simulated. It contains
two keywords, {\ttfamily LEVEL} and {\ttfamily DECAY}. They are defined as follows:

\begin{itemize}
\item LEVEL $N_{Level}$ $P_{i}$ $E_{Level}$ $T_{1/2}$\hfill{} \linebreak
 $N_{Level}$ is the identifier of the level, $P_{i}$ is the relative initial probability 
  of the ejectile (fragment) to end
  in that state after the initial reaction, $E_{Level}$ is the energy of the level 
  above the ground state in keV, and $T_{1/2}$ is the half-life of the level given in ps.
\item DECAY $L_{i}$ $L_{f}$ $P_{\gamma}$\hfill{} \linebreak
  This keyword controls the decay scheme of a initial level $L_{i}$ into the level $L_{f}$ via the relative probability
  $P_{\gamma}$. Note that any given level may possess up to a maximum five levels it decays into.
\end{itemize}

A possible level scheme with two excited levels at $L_{1}= 1000$ keV and $L_{2}= 1500$ keV and an initial level 
probability of $P_{1}= 67$ and $P_{2}= 33$ would look as:\hfill{}
\linebreak
\linebreak
{\ttfamily
  LEVEL  0  0.  0. 0.\linebreak
  LEVEL  1  67. 1000. 10. \linebreak
  LEVEL  2  33. 1500. 20.\linebreak
  DECAY  1  0 100.\linebreak
  DECAY  2  1 100.\linebreak
  DECAY  2  0   1.\linebreak
}
\linebreak
\linebreak
In the above example, $L_{1}$ has a half-life of $T_{1/2} = 10$ ps, while $L_{2}$ has 20 ps. Furthermore, $L_{2}$
decays into the ground-state (LEVEL 0) with a relative probability of 1/100 compared to the decay to 
the first excited state (LEVEL 1).  

\subsection{Input of the $dEdX$ Tables}

The energy loss tables have the structure: \hfill{} \linebreak
\linebreak {\ttfamily dEdX Energy SpecificEnergyLoss}. \hfill{} \linebreak \linebreak
The energy is given in 
MeV/u and the specific energy loss is given in MeV/(mg/cm$^2$).   


\subsection{Including atomic background into the simulation}
\label{sec:abkg}

The program abkg can be used to include atomic background into the simulation.
It can be found under abkg/abkg and requires one input file. Examples input files
are given under abkg/exampleFiles.
As this program produces an hbk-file, you will have to use h2root to convert it to
a root-file. 

For the input keyword ATOMICBG the spectrum name to be specified is h1. The total 
cross-section is given in the title of the spectrum. The total number of $gamma$-rays
per incident event has to be calculated according to the chosen target thickness.

Without ATOMICBG option, it is assumed that the reaction probability per incident
particle is always 100 \%. Now, with ATOMICBG option, the initial level probabilities
of the input level scheme are given in mbarn.

\section{Starting the Simulation}
To start the simulation in the batch mode, type:\hfill{}
\linebreak \linebreak
{\ttfamily
EventGenerator run\_noting.mac
}
\linebreak \linebreak
Omitting a file-name will bring the Event Generator into the ``standard'' GEANT4 interactive session, which is not 
intended to be used for this step.

\chapter{The Event Builder}

The Event Builder is located in the directory {\ttfamily EventBuilderRIKEN}. As mentioned already for the 
Event Generator, make sure that the variables {\ttfamily G4INSTALL}
and {\ttfamily CLHEP\_BASE\_DIR} are set correctly in the file {\ttfamily GNUMakefile}.
Compile the Event Builder with:\hfill{}
\linebreak
\linebreak
{\ttfamily make all}
\linebreak
\linebreak

\section{The Input File(s)}

To change the parameters of the Event Builder, open the file 
{\ttfamily ./input/EventBuilder.in}. The 
parameters of this file are listed in Tab.\ref{tab:EVENTBUILDERKEY}.
The restriction for the Event Generator input file is valid here, too, causing 
the program to terminate if 
comments and not existing input keywords are written in this file (before the END statement).

\begin{table}
  \centering
  \label{tab:EVENTBUILDERKEY}
  %\begin{ruledtabular}
  \begin{tabular}{|l||c|}
    \hline
    Keyword & Parameters \\
    \hline
    \hline
    INPUTFILE                        & $s$             \\
    OUTPUTFILE                       & $s$             \\
    SHIELD                           & $f$ $f$ $f$     \\
    DALI2INCLUDE                     & $i$             \\
    ZPPOSSHIFT                       & $f$             \\
    DALI2FIINCLUDE                   & $i$             \\
    SHOGUNINCLUDE                    & $i$             \\
    SHOGUNFIINCLUDE                  & $i$             \\
    SHOGUNHOUSINGTHICKNESSZYZ        & $i$ $f$ $f$ $f$ \\
    SHOGUNMGOTHICKNESSZYZ            & $i$ $f$ $f$ $f$ \\
    GRAPEINCLUDE                     & $i$             \\
    SGTINCLUDE                       & $i$             \\
    SPHEREINCLUDE                    & $i$ $f$ $f$     \\
    DALI2ENERGYRESOLUTION            & $i$ $f$ $f$     \\
    DALI2ENERGYRESOLUTIONINDIVIDUAL  & $i$             \\
    SHOGUNENERGYRESOLUTION           & $i$ $f$ $f$     \\
    GRAPEENERGYRESOLUTION            & $i$ $f$ $f$     \\
    SGTENERGYRESOLUTION              & $i$ $f$ $f$     \\
    DALI2TIMERESOLUTION              & $f$ $f$         \\
    SHOGUNTIMERESOLUTION             & $f$ $f$         \\
    GRAPETIMERESOLUTION              & $f$ $f$         \\
    SGTTIMERESOLUTION                & $f$ $f$         \\
    POSDETECTORONTARGETRESOLUTION    & $f$             \\
    ENERGYDETECTORAFTERTARGETINCLUDE & $f$             \\
    POSDETECTORAFTERTARGETDISTANCE   & $f$             \\
    POSDETECTORAFTERTARGETRESOLUTION & $f$             \\
    BETARESOLUTION                   & $f$             \\
    BEAMPIPEINCLUDE                  & $i$             \\
    TARGETHOLDERINCLUDE              & $i$             \\
    STQINCLUDE                       & $i$             \\
    COLLIMATORINCLUDE                & $i$             \\
    END                              &                 \\
    \hline
  \end{tabular}
\end{table}

The parameters are now discussed subsequently (if not specified differently, 
the units are cm and degrees, respectively):\hfill{}
\begin{itemize}
\item INPUTFILE FILE$_{in}$ \hfill{} \linebreak
  Gives the location of the input root file generated by the Event Generator.
\item OUTPUTFILE FILE$_{out}$ \hfill{} \linebreak
  Gives the location of the output root file generated in this step.
\item SHIELD $r$ D$_{Pb}$ D$_{Sn}$ \hfill{} \linebreak 
  
  Specifies the thickness of absorber material placed along beam pipe. 
  $r$ is the inner radius of the absorber tube. D$_{Pb}$ D$_{Sn}$ the
  thickness in cm of the Pb and Sn.

\item DALI2INCLUDE i\hfill{} \linebreak
  SHOGUNINCLUDE i\hfill{} \linebreak
  GRAPEINCLUDE i\hfill{} \linebreak
  SGTINCLUDE i\hfill{} \linebreak
  SPHEREINCLUDE i R$_{out}$ R$_{in}$ \hfill{} \linebreak
  If i=1, these detector arrays are included in the simulation. They require an input file that 
  specifies their position and rotation relative to the target. These input files are covered in section
  \ref{Geometry}. For the sphere, the center is (0,0,0) and thickness is determined by R$_{out}$ - R$_{in}$. 
\item DALI2FIINCLUDE i\hfill{} \linebreak
  SHOGUNFIINCLUDE i\hfill{} \linebreak
  If i=1, the first interaction point of a $\gamma$-ray is determined for every crystal of the DALI2/SHOGUN array. Note that for
  cascade decays, every $\gamma$-ray is treated independently. Therefore, if subsequent decays are 
  registered in the same detector, also two first interaction points are determined.
\item ZPOSSHIFT f\hfill{} \linebreak
  Specifies how many cm the detectors are shifted relative to the target. Positive values correspond to an upstream
  shift, negative to a downstream shift.
\item SHOGUNHOUSINGTHICKNESSXYZ i SIZE$_{x}$ SIZE$_{y}$ SIZE$_{z}$\hfill{} \linebreak
  If i=1, the SHOGUN detectors are surrounded by an Al-frame of the thickness SIZE$_{x}$ SIZE$_{y}$ SIZE$_{z}$. 
\item SHOGUNMGOTHICKNESSXYZ i SIZE$_{x}$ SIZE$_{y}$ SIZE$_{z}$\hfill{} \linebreak
  Thickness of the material between housing and crystal, assumed to be MgO
\item DALI2ENERGYRESOLUTION i a b\hfill{} \linebreak
  SHOGUNENERGYRESOLUTION i a b\hfill{} \linebreak
  GRAPEENERGYRESOLUTION i a b\hfill{} \linebreak
  SGTENERGYRESOLUTION i a b\hfill{} \linebreak
  If i=1, the energy resolution has the form: $\Delta E (FWHM) = a + bx$ keV. If i=2, the form is: $\Delta E (FWHM) = ax^{b}$ keV.
\item DALI2ENERGYRESOLUTIONINDIVIDUAL i\hfill{} \linebreak
  If i=1, the energy resolution is read from the file ./input/Dali2Resolution.txt. The global
  resolution given in DALI2ENERGYRESOLUTION is overwritten.
\item DALI2TIMERESOLUTION a b\hfill{} \linebreak
  SHOGUNTIMERESOLUTION a b\hfill{} \linebreak
  GRAPETIMERESOLUTION a b\hfill{} \linebreak
  SGTTIMERESOLUTION a b\hfill{} \linebreak
 The time resolution has the form: $\Delta T (FWHM) = a + bx$ ns, where x is the detected energy in keV.
\item POSDETECTORONTARGETRESOLUTION x\hfill{} \linebreak
  This keywords determines the precision of the tracking onto the target position from imaginary detectors. x is 
  given in cm (FWHM).
\item ENERGYDETECTORAFTERTARGETINCLUDE 0\hfill{} \linebreak
  If i=1, energy detectors after the target will be inserted and included in the simulation. This option is, however,
  not really integrated into the simulation, yet.
\item POSDETECTORAFTERTARGETDISTANCE a\hfill{} \linebreak
  POSDETECTORAFTERTARGETRESOLUTION b\hfill{} \linebreak
  The distance a (in cm) of a position sensitive detector after the secondary target and its resolution b in x and 
  y.
\item BETARESOLUTION a \hfill{} \linebreak
  The $\beta$-resolution a~$=\Delta \beta / \beta (FWHM)$ for the time-of-flight measurement before the target.
  This parameter is necessary for event-by-event Doppler correction of the emitted $\gamma$-rays based on the
  particles' velocities.
\item BEAMPIPEINCLUDE i\hfill{} \linebreak
  If i=1, the beam pipe will be included in the simulation.
\item TARGETHOLDERINCLUDE i\hfill{} \linebreak
  If i=1, the target-holder will be included in the simulation. This option requires to insert the target 
  position, material and thickness in the file:
\linebreak
\linebreak
{\ttfamily ./input/TargetHolder.in}
\linebreak
\linebreak
 The file has the format:\hfill{}
\linebreak
\linebreak
{\ttfamily 0 SetTargetPosition X}\linebreak
{\ttfamily P MATERIAL THICKNESS\linebreak
.\linebreak
.\linebreak
.\linebreak}
\linebreak
\linebreak
Here, {\ttfamily X} determines the set target position of the target holder, {\ttfamily P} the position of the 
to be specified next, {\ttfamily MATERIAL} the material name (see file {\ttfamily ./src/MaterialList.cc} for the
material definitions), and {\ttfamily THICKNESS} the thickness in cm.
\item STQINCLUDE i\hfill{} \linebreak
  If i=1, the STQ17 triplet will be put after the target. Just for optical reasons at the moment.
\item END  \hfill{} \linebreak
  This keyword will end the scan of the input file and can be set at any line within the input file.
\end{itemize}

\section{The Geometry Input Files}
\label{Geometry}

The geometries of the $\gamma$-ray arrays are defined in the directory {\ttfamily ./geometry}.
These input files are slightly different for all the arrays. Therefore,
they are covered independently. If not specified differently, the units are cm and degrees.

The geometry input files for the arrays are given under:\hfill{} 
\linebreak
\linebreak
{\ttfamily ./geometry/dali2\_geometry\_in.txt}\linebreak
{\ttfamily ./geometry/shogun\_geometry\_in.txt}\linebreak
{\ttfamily ./geometry/grape\_geometry\_in.txt}\linebreak
{\ttfamily ./geometry/sgt\_geometry\_in.txt}\linebreak
\linebreak
\linebreak

All input files must end wit a -1 in the last line.
To ensure that your detectors have been put to the correct position, you can check the respective 
{\ttfamily *out.txt} file in the same directory. It has the format 
\linebreak
\linebreak
{\ttfamily ID (TYPE) THETA PHI RADIUS }\linebreak
\linebreak
\linebreak
The parameter {\ttfamily TYPE} is printed only for the DALI2 array.

\subsection{The DALI2 Geometry Input File}

For the DALI2 array, the geometry input file has the form:\hfill{} 
\linebreak
\linebreak
{\ttfamily POSX POSY POSZ PSI THETA PHI ROTSIGN DETTYPE}\linebreak
{\ttfamily .}\linebreak
{\ttfamily .}\linebreak
{\ttfamily .}\linebreak
\linebreak
\linebreak
The crystals' positions are not centered within their outer housing, but shifted by 0.535~cm and 0.7~cm, respectively,
depending on their type. 
{\ttfamily ROTSIGN} defines in which direction they are shifted (+1,-1, 0 for no shift). Furthermore, three different
types of DALI2 crystals exist, which are covered in Ref.\cite{takeuchi:2014:nima}. Therefore, {\ttfamily DETTYPE} specifies
the type of crystal at the given position is to be simulated.

\subsection{The SHOGUN Geometry Input File}

For the DALI3 array, the geometry input file has the form:\hfill{} 
\linebreak
\linebreak
{\ttfamily ID RING TYPE X Y Z RADIUS THETA PHI SIZEX2 SIZEX1 SIZEY2 SIZEY1 SIZEZ}\linebreak
{\ttfamily .}\linebreak
{\ttfamily .}\linebreak
{\ttfamily .}\linebreak
\linebreak
\linebreak
The input values {\ttfamily ID, RING, TYPE,} and {\ttfamily RADIUS} stem from Heiko Scheit's 
script {\ttfamily det\_place.gawk} and are not used for the placement of the detectors. The crystals are of trapezoidal 
shape and defined by the last 5 input values. Example configurations can be found under  
{\ttfamily ./geometry/configurations/*.txt}.

\subsection {{\ttfamily det\_place.gawk}}
The script {\ttfamily det\_place.gawk} is used to calculate sample configurations.
One can find some example commands under {\ttfamily det\_place/ListOfCases.txt}.
The input parameters are:
\begin{itemize}
\item -a \hfill{} \linebreak
  Determines the starting angle of the array
\item -3, -4, etc.\hfill{} \linebreak
  The Doppler broadening of the detectors at $\beta$=0.43.
\item -g Number1 Number2\hfill{} \linebreak
  Number1 is starting angle. Number2 defines how many crystal should be put into one common housing.
\item -t Number\hfill{} \linebreak
Defines the detector geometry specified in the script.
\item Number\hfill{} \linebreak
  Defines how many crystals can be used at maximum.
\item -w Number\hfill{} \linebreak
  How much space has the scrip to leave between the different single-, double-, triple-detectors.
\end{itemize}


\subsection{The Grape Geometry Input File}

The Grape detectors are placed according to:\hfill{}
\linebreak
\linebreak
{\ttfamily X Y Z PSI THETA PHI}\linebreak
{\ttfamily .}\linebreak
{\ttfamily .}\linebreak
{\ttfamily .}\linebreak
\linebreak
\linebreak

No further explanation is necessary, I think.

\subsection{The SGT Geometry Input File}

The Grape detectors are placed according to:\hfill{}
\linebreak
\linebreak
{\ttfamily X Y Z PHI}\linebreak
{\ttfamily .}\linebreak
{\ttfamily .}\linebreak
{\ttfamily .}\linebreak
\linebreak
\linebreak

No further explanation is necessary, I think.

\section{Starting the Simulation}
To start the simulation in the batch mode, type:\hfill{}
\linebreak
\linebreak
{\ttfamily
EventBuilder run\_noting.mac
}
\linebreak
\linebreak
Omitting a file-name will bring the Event Builder into the ``standard'' GEANT4 interactive session. 
To get a view of your detector system, type:\hfill{}
\linebreak
\linebreak
{\ttfamily
/vis/viewer/flush
}
\linebreak
\linebreak
in this session. This will open up the DAWN GUI, from which you can select the viewing point, distance, 
light position, etc... Pressing the ``OK'' button will start the drawing.


\chapter{The Reconstructor}  

The Reconstructor performs the Doppler correction of the simulated $\gamma$-rays.
Examples given in the package are the ROOT-macros
{\ttfamily ShogunReconstructor.C} and {\ttfamily Dali2Reconstructor.C}. Users are 
encouraged to modify the reconstructors as necessary.
A shared library can be created with the command:\hfill{}
\linebreak \linebreak
{\ttfamily root [0] .L ShogunReconstructor.C+
}
\linebreak
\linebreak
from the Root command prompt. Afterwards the Reconstructor can be run with the command:\hfill{}
\linebreak
\linebreak
{\ttfamily root [1] ShogunReconstructor()
}
\linebreak
\linebreak

Before running it, you must specify the parameters from the respective {\ttfamily *.in} input
file, given in Tab.~\ref{tab:RECONSTRUCTORKEY}.

\begin{table}
  \centering
  \label{tab:RECONSTRUCTORKEY}
  %\begin{ruledtabular}
  \begin{tabular}{|l||c|}
    \hline
    Keyword & Parameters \\
    \hline
    \hline
    INPUTFILE                        & $s$             \\
    OUTPUTFILE                       & $s$             \\
    SPECTRABINANDRANGE               & $i$ $f$ $f$     \\
    BETADOPPLERAVERAGE               & $f$             \\
    BETATOFAVERAGE                   & $f$             \\
    DECAYPOSITIONZ                   & $f$             \\
    STATISTICSREDUCTIONFACTOR        & $f$             \\
    FIFIND                           & $i$             \\
    DALI2INCLUDE                     & $i$             \\
    SHOGUNINCLUDE                    & $i$             \\
    GRAPEINCLUDE                     & $i$             \\
    SGTINCLUDE                       & $i$             \\
    ADDBACK                          & $i$ $f$         \\
    TRIGGER                          & $i$             \\ 
    ENERGYTHRESHOLD                  & $i$             \\  
    DETECTORLIMIT                    & $i$             \\
    END                              &                 \\
    \hline
  \end{tabular}
\end{table}

\begin{itemize}
\item INPUTFILE FILE$_{in}$ \hfill{} \linebreak
  Gives the location of the input root file generated by the Event Generator.
\item OUTPUTFILE FILE$_{out}$ \hfill{} \linebreak
  Gives the location of the output root file generated in this step.
\item DALI2INCLUDE i\hfill{} \linebreak
  SHOGUNINCLUDE i\hfill{} \linebreak
  GRAPEINCLUDE i\hfill{} \linebreak
  SGTINCLUDE i\hfill{} \linebreak
  SPHEREINCLUDE i\hfill{} \linebreak
  If i=1, these detector arrays are included in the simulation. 
  In the present script, only the SHOGUN data are analyzed.
\item FIFIND i\hfill{} \linebreak
  If i=1, the average first interaction point of a full energy peak $\gamma$-ray (with fold=1) is determined.
  If i=2, these FI points are used for the Doppler correction. 
\item BETADOPPLERAVERAGE f \hfill{} \linebreak
  The average $\beta$-value used for the Doppler correction.
\item BETATOFAVERAGE f \hfill{} \linebreak
  The average $\beta$-value in front of the target. This value is necessary for an event-by-event
  Doppler correction with different incoming velocities.
\item DECAYPOSITIONZ f\hfill{} \linebreak
  The average z-positon along the beam-axis shifts as a function of the excited states' lifetimes. This
  value can be inserted to correct for this effect.
\item STATISTICSREDUCTIONFACTOR f\hfill{} \linebreak
  If you have simulated many events in the second step and want to see how the Doppler corrected response might
  have looked like for limited statistics you can reduce your statistics by putting a value a $\ge 1$.
\item ADDBACK i f\hfill{} \linebreak
  Used in the Dali2Reconstructor. If i=1, add-back is employed. The second parameter
  specifies the maximum distance in cm between any two detectors for add-back reconstruction. 
\item TRIGGER i\hfill{} \linebreak
  Used in the Dali2Reconstructor. If i=1, it gives the trigger probability as function of $\gamma$-ray
  energy.
\item ENERGYTHRESHOLD f\hfill{} \linebreak
  Used in the Dali2Reconstructor. Defines the minimum energy in keV for add-back reconstruction.
\item DETECTORLIMIT i\hfill{} \linebreak
  Used in the Dali2Reconstructor. i defines the detector id counted as backward/forward detectors
  for certain histograms.
\item END  \hfill{} \linebreak
  This keyword will end the scan of the input file and can be set at any line within the input file.
\end{itemize}

The Doppler corrected spectra will be stored in a ROOT-file, which you can inspect using the TBrowser. 
Type:\hfill{}
\linebreak
\linebreak
{\ttfamily root [2] TBrowser b;
}
\linebreak
\linebreak
and open the folder ``ROOT Files''. It will include the input as well as the output files you used for your 
event reconstruction.

\chapter{Simulation Examples}

\section{DALI2 Efficiency from a $^{60}$Co source}

\subsection{Running the Event Generator}
Your input file {\ttfamily ./EventGenerator/input/EventGenerator.in} should have the following structure:\hfill{}
\linebreak
\linebreak
{\ttfamily %\small
  GAMMAINPUT ./input/60Co.in\linebreak
  NUMBEROFEVENTS 100000\linebreak
  OUTPUTFILE ../tutorial/60CoGenerator.root\linebreak
  END
}
\linebreak
\linebreak
The $\gamma$-ray decay file should be {\ttfamily ./EventGenerator/input/60Co.in} and have the following structure:\hfil{}
\linebreak
\linebreak
{\ttfamily
  LEVEL  0  00.00 0000.000 0.00\linebreak
  LEVEL  1  00.12 1332.510 0.90\linebreak
  LEVEL  2  00.00 2158.610 0.00\linebreak
  LEVEL  3  99.88 2505.748 0.30\linebreak
  DECAY  1  0 99.9826\linebreak
  DECAY  2  1 00.0076\linebreak
  DECAY  2  0 00.0012\linebreak
  DECAY  3  2 00.0075\linebreak
  DECAY  3  1 99.85\linebreak
  DECAY  2  0 00.0000020\linebreak
}
\linebreak
Go into the directory {\ttfamily EventGenerator} and type\hfill{}
\linebreak
\linebreak
{\ttfamily
  EventGenerator run\_nothing.mac
}
\linebreak
\linebreak
into the command line.
This will start the Event Generator and create the root output file {\ttfamily ./tutorial/60CoGenerator.root}.

\subsection{Running the Event Builder}
Your input file {\ttfamily ./EventBuilderRIKEN/input/EventBuilder.in} should have the following structure:\hfill{}
\linebreak
\linebreak
{\ttfamily %\small
  INPUTFILE ../tutorial/60CoGenerator.root\linebreak
  OUTPUTFILE ../tutorial/60CoBuilder.root\linebreak
  SHOGUNINCLUDE 1\linebreak
  SHOGUNENERGYRESOLUTION 2 0.7 0.5\linebreak
  END
}
\linebreak
\linebreak
go into the directory {\ttfamily EventBuilderRIKEN} and type\hfill{}
\linebreak
{\ttfamily
  EventBuilder run\_nothing.mac
}
\linebreak
\linebreak
into the command line. 
This will run the Event Builder and create the root output file {\ttfamily ./tutorial/60CoBuilder.root}.

\subsection{Running the Reconstructor}
Your input file {\ttfamily .Reconstructor/input/Dali2Reconstructor.in} should have the following structure:\hfill{}
\linebreak
\linebreak
{\ttfamily %\small
  INPUTFILE ../tutorial/60CoBuilder.root\linebreak
  OUTPUTFILE ../tutorial/60CoReconstructor.root\linebreak
  SPECTRABINANDRANGE 400 0. 4000.\linebreak
  END
}
\linebreak
\linebreak
Go into the directory {\ttfamily Reconstructor} and open a ROOT-session by typing {\ttfamily root}
into the shell. You have to load/compile the library by typing:\hfill{}
\linebreak
\linebreak
{\ttfamily
 root[].L ShogunReconstructorSimple.C+\linebreak
}
\linebreak
The reconstruction process is started by typing:\hfill{}
\linebreak
\linebreak
{\ttfamily
root[]ShogunReconstructorSimple()\linebreak
}
\linebreak
You can take a look into the created spectra with the TBrowser by typing:\hfill{}
\linebreak
\linebreak
{\ttfamily
root[]TBrowser b;\linebreak
}
\linebreak


\chapter{To-Do List}

It is understood that the simulation has room for improvements. The subsequent list enumerates objects
and features that need to be worked on:

\begin{itemize}

\item The longitudinal momentum spread from the fragmentation is included (Goldhaber formula~\cite{GOLDHABER}), however,
the perpendicular momentum spread is not. Instead, a $\vartheta$-spread has to be given by the input file.
\item The Angular straggling from the energy loss of the heavy ion in the target is not included.
\item There is no $\gamma$-$\gamma$ correlation for anisotropic angular distributions
\item A routine that checks if the detectors overlap has to be implemented.
\end{itemize}
  
\bibliographystyle{unsrt}  
\bibliography{bib}

\end{document}
